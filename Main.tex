\documentclass[12pt]{book}
\usepackage{amsfonts}
\usepackage{amsmath}      % Para \numberwithin
\usepackage{amsthm} 
\usepackage{hyperref}
\usepackage{url}
\theoremstyle{plain}
\numberwithin{equation}{section} %change this to make globally numbered equations
\newtheorem{thm}{Theorem}[section] %remove [section] to make globally numbered environments
\usepackage[spanish]{babel}
\newtheorem{theorem}[thm]{Theorem}
\newtheorem{lemma}[thm]{Lemma}
\newtheorem{example}[thm]{Example}
\newtheorem{definition}[thm]{Definición}
\newtheorem{proposition}[thm]{Proposicion}
\newtheorem{corollary}[thm]{Corollary}
\newtheorem{remark}[thm]{Observación}


% Enumeración 
\renewcommand{\thechapter}{\Roman{chapter}}  % Romana para los capítulos

%Las secciones y subsecciones en números arábigos
\renewcommand{\thesection}{\arabic{section}}
\renewcommand{\thesubsection}{\thesection.\arabic{subsection}}

% Comandos útiles
\newcommand{\deftext}[1]{\textbf{\textit{#1}}}
%%%%%%%%%%%%%%%%%%%%%%%%%%%%%%%%%%%%%%%%%%%%%%%%%%%%%%

\begin{document}
\author{Sebastián Rodríguez Labastida, Álvaro Matanzo Hermoso}
\title{Análisis Topológico de Datos}
\date{\today}
\frontmatter
\maketitle
\tableofcontents
\mainmatter
\chapter{Preludio de Algebra}
\section{Grupos}
\begin{definition}\label{def:grupo}
Un \textit{Grupo}\footnote{Por convención, en este texto todos los grupos serán abelianos.} es un conjunto $A$ junto con una operación $+\colon A\times A\rightarrow A$ que satisface que para cualesquiera $a,b,c \in A$ se cumple
\begin{enumerate}
	\item $a+(b+c) \ =\ (a+b)+c$.
	\item $a+b \ =\ b+a $.
	\item Existe $0_A \in A$ que satisface que para toda $z\in A$ se cumple que $z+0_A \ =\ a$.
	\item Existe $x\in A$ tal que $a+x \ =\ 0_A$.
\end{enumerate}
Cuando no exista ambiguedad usaremos el símbolo $0$ sin mencionar a que grupo pertenece.
\end{definition}

\chapter{Complejos de Cadenas}
\begin{definition}[Complejo de Cadenas]\label{def:CC}
Un \deftext{complejo de cadenas} es una familia $\mathcal{K} :=\{ (K_n, \partial_n) : n\in \mathbb{Z}\}$ tal que para toda $n \in \mathbb{Z}$ se satisfacen las siguientes condiciones
\begin{enumerate}
	\item $K_n$ es un grupo abeliano;
	\item $\partial_n \colon K_{n+1} \to K_n$ es morfismo de grupos abelianos; y
	\item $\partial_n \partial_{n+1} = 0$\label{eq:partC}. 
\end{enumerate}
\end{definition}
\begin{remark}
La ecuación última ecuación es equivalente a decir que para toda $n\in \mathbb{Z}$ se cumple que
\begin{align*}
	\operatorname{im}(\partial_{n+1} ) \leq \ker(\partial_n).
\end{align*}
\end{remark}
\begin{definition}
Dado un complejo de cadenas $\mathcal{K}$ definimos la \deftext{Homología}} de $\mathcal{K}$, en símbolos $H(\mathcal{K})$, como la familia de grupos abelianos $\{ H_n(\mathcal{K}}) \mid n \in \mathbb{Z} \}$, donde
\begin{center}
	$H_n(\mathcal{K})\ :=\ \ker(\partial_n) / \operatorname{im}(\partial_{n+1})$.
\end{center} 
Nos referimos a los elementos de $K_n$ como \deftext{$n$-cadenas}, a los de $\ker(\partial_n)$ como \deftext{$n$-cíclos} y a los de $\operatorname{im}(\partial_n)$ como \deftext{$n$-fronteras}. Cuando no exista confusión escribiremos la clase lateral $c + \operatorname{im}(\partial_{n+1}) \in H_n(\mathcal{K})$ como $[c]$. Además, si $c' \in [c]$, diremos que $c'$ y $c$ son \deftext{homólogos}.
\end{definition}

\begin{definition}[Morfismo de cadenas]
Dados dos complejos de cadenas $\mathcal{K}$ y $\mathcal{K}'$ un \deftext{morfismo de cadenas} $f$, denotado por $f \colon \mathcal{K}\to\mathcal{K}'$ es una familia de morfismos $\{f_n \colon K_n \to K_n' \mid n\in \mathbb{Z}\}$ de tal manera que $\partial_{n+1}' f_{n+1} = f_n \partial_n$.
\end{definition}

\begin{proposition}
\begin{enumerate}
	\item El morfismo trivial de grupos\footnote{Aquí nos referimos al morfismo que manda todo elemento de un grupo al $0$ de otro.} abelianos induce de manera natural un morfismo de cadenas.
	\item Existe $1_\mathcal{K}:\mathcal{K}\to \mathcal{K}$, con $1_{\mathcal{K}_n} = \text{Id}_{K_n}$ la identidad.
	\item S $\mathcal{K}$, $\mathcal{J}$ y $\mathcal{L}$ son complejos de cadenas y $f\colon \mathcal{K}\to \mathcal{J}$ y $g \colon \mathcal{J}\to \mathcal{L}$ son morfismos de cadenas, entonces la familia de morfismos $\{ g_n \circ f_n : n\in \mathbb{Z} \}$ es el morfismo de cadenas $g\circ f \colon \mathcal{K}\to \mathcal{L}$.
\end{enumerate}
2 y 3 en suma nos dicen que existe la categoría de complejos de cadenas. 
\end{proposition}
\begin{proposition}
Dados dos complejos de cadenas $\mathcal{K}$ y $\mathcal{L}$ y un morfismo de cadenas $f:\mathcal{K} \to \mathcal{L}$, se tiene que la familia de morfismos $A$, $\{H_n(f) : n\in \mathbb{Z} \}$ donde el morfismo $H_n(f):H_n(\mathcal{K})\to H_n(\mathcal{L})$ está dado por
\begin{center}
$H_n(f)(c + \operatorname{im}\partial_{n+1}) = f(c) + \operatorname{im}\partial_{n+1}$,
\end{center}
es un morfismo de grupos abelianos. 
\end{proposition}

\end{document}