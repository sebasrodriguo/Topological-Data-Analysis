\documentclass[12pt]{book}
\usepackage{amsfonts}
\usepackage{amsmath}      % Para \numberwithin
\usepackage{amsthm} 
\usepackage{hyperref}
\usepackage{url}
\theoremstyle{plain}
\numberwithin{equation}{section} %change this to make globally numbered equations
\newtheorem{thm}{Theorem}[section] %remove [section] to make globally numbered environments
\usepackage[spanish]{babel}
\newtheorem{theorem}[thm]{Theorem}
\newtheorem{lemma}[thm]{Lemma}
\newtheorem{example}[thm]{Example}
\newtheorem{definition}[thm]{Definición}
\newtheorem{proposition}[thm]{Proposition}
\newtheorem{corollary}[thm]{Corollary}

% Enumeración 
\renewcommand{\thechapter}{\Roman{chapter}}  % Romana para los capítulos

%Las secciones y subsecciones en números arábigos
\renewcommand{\thesection}{\arabic{section}}
\renewcommand{\thesubsection}{\thesection.\arabic{subsection}}

%%%%%%%%%%%%%%%%%%%%%%%%%%%%%%%%%%%%%%%%%%%%%%%%%%%%%%

\begin{document}
\author{Sebastián Rodríguez Labastida, Álvaro Matanzo Hermoso}
\title{Análisis Topológico de Datos}
\date{\today}
\frontmatter
\maketitle

\chapter{Preludio de Algebra}
\section{Grupos}
\begin{definition}\label{def:grupo}
Un \textit{Grupo}\footnote{Por convención, en este texto todos los grupos serán abelianos.} es un conjunto $A$ junto con una operación $+\colon A\times A\rightarrow A$ que satisface que para cualesquiera $a,b,c \in A$ se cumple
\begin{enumerate}
	\item $a+(b+c) \ =\ (a+b)+c$.
	\item $a+b \ =\ b+a $.
	\item Existe $0_A \in A$ que satisface que para toda $z\in A$ se cumple que $z+0_A \ =\ a$.
	\item Existe $x\in A$ tal que $a+x \ =\ 0_A$.
\end{enumerate}
Cuando no exista ambiguedad usaremos el símbolo $0$ sin mencionar a que grupo pertenece.
\end{definition}

% ============================================================
% Definiciones para issue #17: grupo abeliano libre, tensor y Tor
% Basadas en Fuchs, Abelian Groups (Springer, 2015)
% ============================================================

\begin{definition}\label{def:free-abelian-group}
Un \textit{grupo abeliano libre} es una suma directa de grupos cíclicos infinitos.
Si estos grupos cíclicos son generados por elementos $x_i$ $(i \in I)$, entonces
el grupo libre será
\[
F = \bigoplus_{i \in I} \langle x_i \rangle.
\]
El conjunto $\{x_i\}_{i \in I}$ es una base de $F$. Los elementos de $F$ son
combinaciones lineales finitas de la forma
\[
g \ =\ n_1x_{i_1} + \cdots + n_kx_{i_k}, \quad \text{con } n_j \in \mathbb{Z},
\]
con coeficientes enteros distintos de cero. Dos combinaciones representan el
mismo elemento de $F$ si y solo si difieren en el orden de los términos.
La suma se define añadiendo los coeficientes de los mismos generadores.
En particular, $F$ está determinado, salvo isomorfismos, por la cardinalidad
de sus generadores, al cual se llama el \textit{rango} del grupo libre.
\end{definition}

\begin{definition}\label{def:tensor-product}
Sean $A$ y $C$ grupos abelianos. Sea $X$ el grupo libre sobre el conjunto
$A \times C$, cuyos generadores son $(a,c)$ con $a \in A$ y $c \in C$.
Sea $Y$ el subgrupo de $X$ generado por todos los elementos de la forma
\[
(a_1 + a_2, c) - (a_1, c) - (a_2, c), \quad
(a, c_1 + c_2) - (a, c_1) - (a, c_2),
\]
para todo $a,a_1,a_2 \in A$ y $c,c_1,c_2 \in C$. El \textit{producto tensorial}
de $A$ y $C$ se define como el cociente
\[
A \otimes C \ =\ X / Y.
\]
Dado $(a,c) \in A \times C$ definimos el tensor de $a$ con $c$, $a \otimes c$, como la clase del generador $(a,c)$; entonces los elementos de
$A \otimes C$ son combinaciones lineales finitas de tales tensores, sujetas a
las relaciones
\[
(a_1 + a_2) \otimes c \ =\ a_1 \otimes c + a_2 \otimes c, \qquad
a \otimes (c_1 + c_2) \ =\ a \otimes c_1 + a \otimes c_2.
\]
El grupo $A \otimes C$ cumple la \textit{propiedad universal}: para toda
función bilineal $g : A \times C \to G$, con $G$ un grupo abeliano, existe un único
morfismo $\overline{g} : A \otimes C \to G$ tal que
$\overline{g}(a \otimes c) = g(a,c)$. Esta propiedad caracteriza a
$A \otimes C$ de manera única, salvo isomorfismos.
\end{definition}

\begin{definition}\label{def:tor}
El \textit{producto de torsión} de dos grupos abelianos $A$ y $C$, denotado por
$\mathrm{Tor}(A,C)$, se define como el grupo abeliano libre generado por las tercias
$(a,m,c)$ con $a \in A$, $c \in C$ y $m \in \mathbb{N}$ tales que $ma = 0 = mc$,
sujetos a las relaciones
\[
\begin{aligned*}
(a_1 + a_2, m, c) &= (a_1, m, c) + (a_2, m, c),\\
(a, m, c_1 + c_2) &= (a, m, c_1) + (a, m, c_2),\\
(a, mn, c) &= (na, m, c) = (a, m, nc).
\end{aligned*}
\]
Con estas relaciones, el grupo $\mathrm{Tor}(A,C)$ es abeliano y satisface una
\textit{simetría natural}:
\[
\mathrm{Tor}(A,C) \cong \mathrm{Tor}(C,A).
\]
El producto de torsión es, por construcción, functorial en ambos argumentos y
mide el ``defecto de exactitud a la izquierda'' del producto tensorial.
\end{definition}

% ============================================================

\end{document}