\documentclass[12pt]{book}
\usepackage{amsfonts}
\usepackage{amsmath}      % Para \numberwithin
\usepackage{amsthm} 
\usepackage{hyperref}
\usepackage{url}
\theoremstyle{plain}
\numberwithin{equation}{section} %change this to make globally numbered equations
\newtheorem{thm}{Theorem}[section] %remove [section] to make globally numbered environments
\usepackage[spanish]{babel}
\newtheorem{theorem}[thm]{Theorem}
\newtheorem{lemma}[thm]{Lemma}
\newtheorem{example}[thm]{Example}
\newtheorem{definition}[thm]{Definición}
\newtheorem{proposition}[thm]{Proposition}
\newtheorem{corollary}[thm]{Corollary}
\begin{document}
\author{Sebastián Rodríguez Labastida, Álvaro Matanzo Hermoso}
\title{Análisis Topológico de Datos}
\date{\today}
\frontmatter
\maketitle

\begin{definition}
Un \textit{Grupo}\footnote{Por convención, en este texto todos los grupos serán abelianos.} es un conjunto $A$ junto con una operación $+ \in A \times A$ que satisface que para cualesquiera $a,b,c \in A$ se cumple
\begin{enumerate}
	\item $a+b\in A $.
	\item $a+(b+c) \ =\ (a+b) + c$.
	\item $a+b \ =\ b+a $.
	\item Existe $x\in A$ tal que $a+x \ =\ a$.
	\item Existe $0_A \in A$ que satisface que para toda $\ell \in A$ se cumple que $a+0 \ =\ a$.
\end{enumerate}
\end{definition}

\end{document}