\documentclass[12pt]{book}
\usepackage{amsfonts}
\usepackage{amsmath}      % Para \numberwithin
\usepackage{amsthm} 
\usepackage{hyperref}
\usepackage{graphicx}
\usepackage{url}
\theoremstyle{plain}
\numberwithin{equation}{section} %change this to make globally numbered equations
\newtheorem{thm}{Theorem}[section] %remove [section] to make globally numbered environments
\usepackage[spanish]{babel}
\newtheorem{theorem}[thm]{Theorem}
\newtheorem{lemma}[thm]{Lemma}
\newtheorem{example}[thm]{Example}
\newtheorem{definition}[thm]{Definición}
\newtheorem{proposition}[thm]{Proposicion}
\newtheorem{corollary}[thm]{Corollary}
\newtheorem{remark}[thm]{Observación}


% Enumeración 
\renewcommand{\thechapter}{\Roman{chapter}}  % Romana para los capítulos

%Las secciones y subsecciones en números arábigos
\renewcommand{\thesection}{\arabic{section}}
\renewcommand{\thesubsection}{\thesection.\arabic{subsection}}
% Comandos útiles
\newcommand{\deftext}[1]{\textbf{\textit{#1}}}
%%%%%%%%%%%%%%%%%%%%%%%%%%%%%%%%%%%%%%%%%%%%%%%%%%%%%%

\begin{document}
\author{Sebastián Rodríguez Labastida, Álvaro Matanzo Hermoso}
\title{Análisis Topológico de Datos}
\date{\today}
\frontmatter
\maketitle
\tableofcontents
\mainmatter
\chapter{Preludio de Algebra}
\section{Grupos}
\begin{definition}\label{def:grupo}
Un \deftext{grupo}\footnote{Por convención, en este texto todos los grupos serán abelianos.} es un conjunto $A$ junto con una operación $+\colon A\times A\rightarrow A$ que satisface que para cualesquiera $a,b,c \in A$ se cumple
\begin{enumerate}
	\item $a+(b+c) \ =\ (a+b)+c$.
	\item $a+b \ =\ b+a $.
	\item Existe $0_A \in A$ que satisface que para toda $z\in A$ se cumple que $z+0_A \ =\ a$.
	\item Existe $x\in A$ tal que $a+x \ =\ 0_A$.
\end{enumerate}
Cuando no exista ambiguedad usaremos el símbolo $0$ sin mencionar a que grupo pertenece.
\end{definition}

\begin{definition}
Sea $\{B_i\}_{i\in I}$ una familia de grupos. Un \emph{vector}
\[
(\ldots, b_i, \ldots)
\]
es una familia que asigna a cada índice $i \in I$ un elemento $b_i \in B_i$.
También puede verse como una función
\[
f : I \longrightarrow \bigcup_{i\in I} B_i
\]
tal que
\[
f(i) = b_i \in B_i \quad \text{para cada } i\in I.
\]

La igualdad y la suma de vectores se definen \emph{coordenada a coordenada}:
\[
(b_i)_{i\in I} = (b_i')_{i\in I}
\quad\Longleftrightarrow\quad
b_i = b_i' \ \text{para todo } i\in I,
\]
\[
(b_i)_{i\in I} + (b_i')_{i\in I}
= (\,b_i + b_i'\,)_{i\in I}.
\]

El conjunto de todos estos vectores, con la operación descrita, forma un grupo que se denota por
\[
C = \prod_{i\in I} B_i,
\]
y se llama el \emph{producto directo} o \emph{producto cartesiano} de los grupos $B_i$.
\end{definition}


\begin{definition}
Sean $\{B_i\}_{i \in I}$ grupos abelianos. Se llama \emph{suma directa} de los grupos $B_i$ al subgrupo del producto directo
\[
\bigoplus_{i \in I} B_i 
= 
\left\{
(b_i)_{i \in I} \in \prod_{i \in I} B_i 
\;\middle|\;
b_i = 0 \text{ salvo para un número finito de índices } i
\right\},
\]
donde la operación de grupo está dada componente a componente.

A los homomorfismos de inclusión
\[
\rho_i : B_i \longrightarrow \bigoplus_{i \in I} B_i,
\qquad
b_i \longmapsto (\ldots,0,b_i,0,\ldots),
\]
se les llama \emph{inyecciones coordenadas}. 

El grupo $\bigoplus_{i \in I} B_i$ se denomina también \emph{suma directa externa} de los $B_i$ y puede verse como el subgrupo del producto directo formado por los elementos con soporte finito.
\end{definition}

% ============================================================
% Definiciones para issue #17: grupo abeliano libre, tensor y Tor
% Basadas en Fuchs, Abelian Groups (Springer, 2015)
% ============================================================

\begin{definition}\label{def:free-abelian-group}
Un \deftext{grupo abeliano libre} es una suma directa de grupos cíclicos infinitos.
Si estos grupos cíclicos son generados por elementos $x_i$ $(i \in I)$, entonces
el grupo libre será
\[
F  \ = \  \bigoplus_{i \in I} \langle x_i \rangle.
\]
El conjunto $\{x_i\}_{i \in I}$ es una \textit{base} de $F$. Los elementos de $F$ son
combinaciones lineales finitas de la forma
\[
g \ =\ n_1x_{i_1} + \cdots + n_kx_{i_k}, \quad \text{con } n_j \in \mathbb{Z},
\]
Dos combinaciones representan el
mismo elemento de $F$ si y solo si difieren en el orden de los términos.
La suma se define añadiendo los coeficientes de los mismos generadores.
En particular, $F$ está determinado, salvo isomorfismos, por la cardinalidad
de sus generadores, al cual se llama el \textit{rango} del grupo libre.
\end{definition}

\begin{theorem}[Propiedad universal de los grupos libres]
Sea $X$ un conjunto libre de generadores del grupo libre $F$. 
Toda función
\[
f : X \longrightarrow A
\]
con valores en un grupo abeliano $A$ se extiende de manera única a un homomorfismo
\[
\varphi : F \longrightarrow A.
\]
Esta propiedad caracteriza a los conjuntos libres de generadores, y por lo tanto a los grupos libres.
\end{theorem}


\begin{definition}\label{def:tensor-product}
Sean $A$ y $C$ grupos abelianos. Sea $X$ el grupo libre sobre el conjunto
$A \times C$, cuyos generadores son $(a,c)$ con $a \in A$ y $c \in C$.
Sea $Y$ el subgrupo de $X$ generado por todos los elementos de la forma
\[
(a_1 + a_2, c) - (a_1, c) - (a_2, c), \quad
(a, c_1 + c_2) - (a, c_1) - (a, c_2),
\]
para todo $a,a_1,a_2 \in A$ y $c,c_1,c_2 \in C$. El \textit{producto tensorial}
de $A$ y $C$ se define como el cociente
\[
A \otimes C \ =\ X / Y.
\]
Dado $(a,c) \in A \times C$ definimos el \textit{tensor} de $a$ con $c$, $a \otimes c$, como la clase del generador $(a,c)$; entonces los elementos de
$A \otimes C$ son combinaciones lineales finitas de tales tensores, sujetas a
las relaciones
\[
(a_1 + a_2) \otimes c \ =\ a_1 \otimes c + a_2 \otimes c, \qquad
a \otimes (c_1 + c_2) \ =\ a \otimes c_1 + a \otimes c_2.
\]
El grupo $A \otimes C$ cumple la \textit{propiedad universal}: para toda
función bilineal $g : A \times C \to G$, con $G$ un grupo abeliano, existe un único
morfismo $\overline{g} : A \otimes C \to G$ tal que
$\overline{g}(a \otimes c) = g(a,c)$. Esta propiedad caracteriza a
$A \otimes C$ de manera única, salvo isomorfismos.
\end{definition}

\begin{definition}
La parte de torsión de un grupo $A$ es
\[
t(A) := \{\, x \in A \mid nx = 0 \text{ para algún } n \in \mathbb{N} \,\}.
\]
Si $t(A)=A$, decimos que $A$ es de torsión.
En este caso,
\[
A = \bigoplus_{p} A_p.
\]
\end{definition}

\begin{theorem}[Teorema fundamental de los grupos finitamente generados]
Todo grupo $G$ tal que existe  $X \in [G]^{<\omega}$\footnote{En general, si $A$ es un conjunto, entonces $[A]^{<\omega}$ representa la colección de todos los subconjuntos finitos de $A$. } con
$\langle X \rangle = G$ es isomorfo a la suma directa de un grupo libre finitamente 
generado y un número finito de grupos cíclicos de orden $p^{k}$, con $p$ primo 
y $k \ge 1$. Esta descomposición es única salvo isomorfismos. En particular,
\[
G = F(G) \oplus t(G),
\]
donde $F(G)$ es la parte libre de $G$ y $t(G)$ es su parte de torsión.
\end{theorem}

\begin{definition}\label{def:tor}
El \deftext{producto de torsión} de dos grupos abelianos $A$ y $C$, denotado por
$\mathrm{Tor}(A,C)$, se define como el grupo abeliano libre generado por las tercias
$(a,m,c)$ con $a \in A$, $c \in C$ y $m \in \mathbb{N}$ tales que $ma = 0 = mc$,
sujetos a las relaciones
\begin{align*}
(a_1 + a_2, m, c) &= (a_1, m, c) + (a_2, m, c),\\
(a, m, c_1 + c_2) &= (a, m, c_1) + (a, m, c_2),\\
(a, mn, c) &= (na, m, c) = (a, m, nc).
\end{align*}
Con estas relaciones, el grupo $\operatorname{Tor}(A,C)$ es abeliano y satisface una
\textit{simetría natural}
\[
\operatorname{Tor}(A,C) \cong \operatorname{Tor}(C,A).
\]
\end{definition}

\begin{proposition}
Sea $\mathcal{G}$ una familia de grupos abelianos. Entonces,
\[
\left(\bigoplus_{G\in\mathcal{G}} G\right) \otimes C 
\;\cong\;
\bigoplus_{G\in\mathcal{G}} (G \otimes C).
\]
\end{proposition}


\begin{proposition}
Si $A$ y $C$ son finitos, entonces
\[
A \otimes C \;\cong\; \operatorname{Tor}(A, C),
\]
y ambos funtores son conmutativos; es decir,
\[
A \otimes C \cong C \otimes A 
\quad\text{y}\quad 
\operatorname{Tor}(A,C) \cong \operatorname{Tor}(C,A).
\]
\end{proposition}


% ============================================================

\chapter{Complejos}
\section{Complejos de Cadenas}
\begin{definition}[Complejo de Cadenas]\label{def:CC}
Un \deftext{complejo de cadenas} es una familia $\mathcal{K} :=\{ (K_n, \partial_n) \colon n\in \mathbb{Z}\}$ tal que para toda $n \in \mathbb{Z}$ se satisfacen las siguientes condiciones
\begin{enumerate}
	\item $K_n$ es un grupo abeliano;
	\item $\partial_n \colon K_{n+1} \to K_n$ es morfismo de grupos abelianos; y
	\item $\partial_n \partial_{n+1} = 0$\label{eq:partC}. 
\end{enumerate}
\end{definition}
\begin{remark}
La ecuación última es equivalente a decir que para toda $n\in \mathbb{Z}$ se cumple que
\begin{align*}
	\operatorname{im}(\partial_{n+1} ) \leq \ker(\partial_n).
\end{align*}
\end{remark}
\begin{definition}
Dado un complejo de cadenas $\mathcal{K}$ definimos la \deftext{Homología} de $\mathcal{K}$, en símbolos $H(\mathcal{K})$, como la familia de grupos abelianos $\{ H_n(\mathcal{K}}) \mid n \in \mathbb{Z} \}$, donde
\begin{center}
	$H_n(\mathcal{K})\ :=\ \ker(\partial_n) / \operatorname{im}(\partial_{n+1})$.
\end{center} 
Nos referimos a los elementos de $K_n$ como \deftext{$n$-cadenas}, a los de $\ker(\partial_n)$ como \deftext{$n$-cíclos} y a los de $\operatorname{im}(\partial_n)$ como \deftext{$n$-fronteras}. Cuando no exista confusión escribiremos la clase lateral $c + \operatorname{im}(\partial_{n+1}) \in H_n(\mathcal{K})$ como $[c]$. Además, si $c' \in [c]$, diremos que $c'$ y $c$ son \deftext{homólogos}.
\end{definition}

\begin{definition}[Morfismo de cadenas]
Dados dos complejos de cadenas $\mathcal{K}$ y $\mathcal{K}'$ un \deftext{morfismo de cadenas} $f$, denotado por $f \colon \mathcal{K}\to\mathcal{K}'$ es una familia de morfismos $\{f_n \colon K_n \to K_n' \mid n\in \mathbb{Z}\}$ de tal manera que $\partial_{n+1}' f_{n+1} = f_n \partial_n$.
\end{definition}

\begin{proposition}
\begin{enumerate}
	\item El morfismo trivial de grupos\footnote{Aquí nos referimos al morfismo que manda todo elemento de un grupo al $0$ de otro.} abelianos induce de manera natural un morfismo de cadenas.
	\item Existe $1_\mathcal{K}:\mathcal{K}\to \mathcal{K}$, con $1_{\mathcal{K}_n} = \text{Id}_{K_n}$ la identidad.
	\item S $\mathcal{K}$, $\mathcal{J}$ y $\mathcal{L}$ son complejos de cadenas y $f\colon \mathcal{K}\to \mathcal{J}$ y $g \colon \mathcal{J}\to \mathcal{L}$ son morfismos de cadenas, entonces la familia de morfismos $\{ g_n \circ f_n \mid n\in \mathbb{Z} \}$ es el morfismo de cadenas $g\circ f \colon \mathcal{K}\to \mathcal{L}$.
\end{enumerate}
2 y 3 en suma nos dicen que existe la categoría de complejos de cadenas. 
\end{proposition}
\begin{proposition}
Dados dos complejos de cadenas $\mathcal{K}$ y $\mathcal{L}$ y un morfismo de cadenas $f:\mathcal{K} \to \mathcal{L}$, se tiene que la familia de morfismos $A$, $\{H_n(f) : n\in \mathbb{Z} \}$ donde el morfismo $H_n(f):H_n(\mathcal{K})\ \to H_n(\mathcal{L})$ está dado por
\begin{center}
$H_n(f)(c + \operatorname{im}\partial_{n+1}) = f(c) + \operatorname{im}\partial_{n+1}$,
\end{center}
es un morfismo de grupos abelianos. 
\end{proposition}


% ============================================================

\chapter{Ejercicios}

\section*{Ejercicio 1}

Calcular la homología de cualquier árbol, es decir, de cualquier gráfica simple conexa sin ciclos.


\section*{Ejercicio 2}

Calcula la homología del plano proyectivo.

Tomamos la triangulación del plano proyectivo descrita por el complejo simplicial

\[
\begin{aligned}
D = \{\,
&\emptyset,\;
v_1, v_2, v_3, v_4, v_5, v_6,\\[4pt]
&v_1v_2,\; v_1v_3,\; v_1v_4,\; v_1v_5,\; v_1v_6,\;
v_2v_3,\; v_2v_4,\; v_2v_5,\; v_2v_6,\;
v_3v_4,\; v_3v_5,\; v_3v_6,\;
v_4v_5,\; v_4v_6,\; v_5v_6,\\[4pt]
&v_1v_2v_4,\; v_1v_2v_6,\; v_1v_4v_5,\; v_1v_5v_6,\; v_1v_6v_2,\;
v_2v_3v_5,\; v_2v_5v_6,\; v_3v_4v_6,\; v_3v_6v_5,\; v_4v_5v_6
\,\}.
\end{aligned}
\]

\noindent
Tomamos ahora las $\partial_n$ y los grupos de cadenas $C_n(D)$ como las definimos anteriormente.
Para obtener el rango y los generadores de los grupos de homología del complejo simplicial $D$
seguiremos el método matricial visto en clase.
Primero representamos matricialmente los morfismos frontera.

\bigskip

% -------------------------------------------------------
\subsection*{$\partial_0$}

Como $\partial_0 : C_0(D) \to C_{-1}(D)$ está dada por $\partial_0(v) = \emptyset$, para $v \in C_0(D)$,

\[
\partial_0(v_1) = \partial_0(v_2) = \partial_0(v_3)
= \partial_0(v_4) = \partial_0(v_5) = \partial_0(v_6) = \emptyset.
\]

Así, obtenemos la matriz $A$:

\[
A =
\begin{pmatrix}
1 & 1 & 1 & 1 & 1 & 1
\end{pmatrix}.
\]

\bigskip

% -------------------------------------------------------
\subsection*{$\partial_1$}

$\partial_1 : C_1(D) \to C_0(D)$ está dada por $\partial_1(vw) = w - v$, para $vw \in C_1(D)$.

\[
\begin{aligned}
\partial_1(v_1v_2) &= v_2 - v_1, &
\partial_1(v_1v_3) &= v_3 - v_1, &
\partial_1(v_1v_4) &= v_4 - v_1,\\[2pt]
\partial_1(v_1v_5) &= v_5 - v_1, &
\partial_1(v_1v_6) &= v_6 - v_1, &
\partial_1(v_2v_3) &= v_3 - v_2,\\[2pt]
\partial_1(v_2v_5) &= v_5 - v_2, &
\partial_1(v_2v_6) &= v_6 - v_2, &
\partial_1(v_3v_4) &= v_4 - v_3,\\[2pt]
\partial_1(v_3v_5) &= v_5 - v_3, &
\partial_1(v_3v_6) &= v_6 - v_3, &
\partial_1(v_4v_5) &= v_5 - v_4,\\[2pt]
\partial_1(v_4v_6) &= v_6 - v_4, &
\partial_1(v_5v_6) &= v_6 - v_5.
\end{aligned}
\]

Obteniendo así la matriz $B$:

\[
B = \small
\begin{pmatrix}
-1 & -1 & -1 & -1 & -1 & 0 & 0 & 0 & 0 & 0 & 0 & 0 & 0 & 0 & 0\\
 1 &  0 &  0 &  0 &  0 & -1 & -1 & -1 & -1 & 0 & 0 & 0 & 0 & 0 & 0 \\
 0 &  1 &  0 &  0 &  0 &  1 &  0 &  0 & 0 & -1 & -1 & -1 & 0 & 0 & 0 \\
 0 &  0 &  1 &  0 &  0 &  0 &  1 &  0 &  0 &  1 & 0 & 0 & -1 & -1 & 0 \\
 0 &  0 &  0 &  1 &  0 &  0 &  0 &  1 &  0 &  0 &  1 & 0 & 1 & 0 & -1 \\
 0 &  0 &  0 &  0 &  1 &  0 &  0 &  0 &  1 &  0 &  0 & 1 & 0 & 1 & 1
\end{pmatrix}.
\]

\normalsize

\bigskip

% -------------------------------------------------------
\subsection*{$\partial_2$}

$\partial_2 : C_2(D) \to C_1(D)$ está dada por 
\[
\partial_2(vwz) = wz - vz + vw.
\]

Escribimos explícitamente:

\[
\begin{aligned}
\partial_2(v_1 v_2 v_4) &= v_2 v_4 - v_1 v_4 + v_1 v_2,\\
\partial_2(v_1 v_2 v_6) &= v_2 v_6 - v_1 v_6 + v_1 v_2,\\
\partial_2(v_1 v_3 v_4) &= v_3 v_4 - v_1 v_4 + v_1 v_3,\\
\partial_2(v_1 v_3 v_5) &= v_3 v_5 - v_1 v_5 + v_1 v_3,\\
\partial_2(v_1 v_5 v_6) &= v_5 v_6 - v_1 v_6 + v_1 v_5,\\
\partial_2(v_2 v_3 v_5) &= v_3 v_5 - v_2 v_5 + v_2 v_3,\\
\partial_2(v_2 v_3 v_6) &= v_3 v_6 - v_2 v_6 + v_2 v_3,\\
\partial_2(v_2 v_4 v_5) &= v_4 v_5 - v_2 v_5 + v_2 v_4,\\
\partial_2(v_3 v_4 v_6) &= v_4 v_6 - v_3 v_6 + v_3 v_4,\\
\partial_2(v_3 v_5 v_6) &= v_5 v_6 - v_3 v_6 + v_3 v_5,\\
\partial_2(v_4 v_5 v_6) &= v_5 v_6 - v_4 v_6 + v_4 v_5.
\end{aligned}
\]

Obteniendo así la matriz $C$:

\[
C = \small
\begin{pmatrix}
 1 &  1 &  0 &  0 &  0 &  0 &  0 &  0 &  0 &  0 \\
 0 &  0 &  1 &  1 &  0 &  0 &  0 &  0 &  0 &  0 \\
-1 &  0 & -1 &  0 &  0 &  0 &  0 &  0 &  0 &  0 \\
 0 &  0 &  0 & -1 &  1 &  0 &  0 &  0 &  0 &  0 \\
 0 & -1 &  0 &  0 & -1 &  0 &  0 &  0 &  0 &  0 \\
 0 &  0 &  0 &  0 &  0 &  1 &  1 &  0 &  0 &  0 \\
 1 &  0 &  0 &  0 &  0 &  0 &  0 &  1 &  0 &  0 \\
 0 &  0 &  0 &  0 &  0 & -1 &  0 & -1 &  0 &  0 \\
 0 &  1 &  0 &  0 &  0 &  0 & -1 &  0 &  0 &  0 \\
 0 &  0 &  1 &  0 &  0 &  0 &  0 &  0 &  1 &  0 \\
 0 &  0 &  0 &  1 &  0 &  1 &  0 &  0 &  0 &  0 \\
 0 &  0 &  0 &  0 &  0 &  0 &  1 &  0 & -1 &  0 \\
 0 &  0 &  0 &  0 &  0 &  0 &  0 &  1 &  0 &  1 \\
 0 &  0 &  0 &  0 &  0 &  0 &  1 &  0 &  1 & -1 \\
 0 &  0 &  0 &  0 &  1 &  0 &  0 &  0 &  0 &  1
\end{pmatrix}.
\]
\normalsize

\bigskip

% -------------------------------------------------------
\subsection*{$\partial_n,\ n>2$}

Por cómo están definidas las $\partial_n$, no es necesario realizar las representaciones
matriciales para saber quiénes son $\ker(\partial_n)$ y $\operatorname{im}(\partial_n)$.

\bigskip

% -------------------------------------------------------
\subsection*{Formas normales de Smith}

Dada una matriz $M'$ en forma normal de Smith, las columnas cuyo pivote es $0$
corresponden a una base del núcleo, y las columnas con pivote generan la imagen.

A continuación presentamos explícitamente las matrices $A', Q_A, B', Q_B, C'$ y $Q_C$.

\bigskip

\paragraph{1. Matriz $A'$ y matriz de cambio de base $Q_A$}

\[
A' =
\begin{pmatrix}
1 & 0 & 0 & 0 & 0 & 0
\end{pmatrix},
\qquad
Q_A =
\begin{pmatrix}
 1 & -1 & -1 & -1 & -1 & -1 \\
 0 &  0 &  0 &  0 &  0 &  1 \\
 0 &  0 &  0 &  0 &  1 &  0 \\
 0 &  0 &  0 &  1 &  0 &  0 \\
 0 &  0 &  1 &  0 &  0 &  0 \\
 0 &  1 &  0 &  0 &  0 &  0
\end{pmatrix}.
\]

\bigskip

\paragraph{2. Matriz $B'$ y matriz $Q_B$}

\[
B' =
\begin{pmatrix}
1 & 0 & 0 & 0 & 0 & 0 & 0 & 0 & 0 & 0 & 0 & 0 & 0 & 0 & 0 \\
0 & 1 & 0 & 0 & 0 & 0 & 0 & 0 & 0 & 0 & 0 & 0 & 0 & 0 & 0 \\
0 & 0 & 1 & 0 & 0 & 0 & 0 & 0 & 0 & 0 & 0 & 0 & 0 & 0 & 0 \\
0 & 0 & 0 & 1 & 0 & 0 & 0 & 0 & 0 & 0 & 0 & 0 & 0 & 0 & 0 \\
0 & 0 & 0 & 0 & 1 & 0 & 0 & 0 & 0 & 0 & 0 & 0 & 0 & 0 & 0 \\
0 & 0 & 0 & 0 & 0 & 0 & 0 & 0 & 0 & 0 & 0 & 0 & 0 & 0 & 0
\end{pmatrix}.
\]

\[
\small
Q_B =
\begin{pmatrix}
1 & 0 & 0 & 0 & 0 & 0 & 0 & 0 & 1 & 0 & 0 & 1 & 0 & 1 & 1 \\
0 & 1 & 0 & 0 & 0 & 0 & 0 & 1 & 0 & 0 & 1 & 0 & 1 & 0 & -1 \\
0 & 0 & 1 & 0 & 0 & 0 & 1 & 0 & 0 & 1 & 0 & 0 & -1 & -1 & 0 \\
0 & 0 & 0 & 1 & 0 & 1 & 0 & 0 & 0 & -1 & -1 & -1 & 0 & 0 & 0 \\
0 & 0 & 0 & 0 & 1 & -1 & -1 & -1 & -1 & 0 & 0 & 0 & 0 & 0 & 0 \\
0 & 0 & 0 & 0 & 0 & 0 & 0 & 0 & 0 & 0 & 0 & 0 & 0 & 0 & 1 \\
0 & 0 & 0 & 0 & 0 & 0 & 0 & 0 & 0 & 0 & 0 & 0 & 0 & 1 & 0 \\
0 & 0 & 0 & 0 & 0 & 0 & 0 & 0 & 0 & 0 & 0 & 1 & 0 & 0 & 0 \\
0 & 0 & 0 & 0 & 0 & 0 & 0 & 0 & 1 & 0 & 0 & 0 & 0 & 0 & 0 \\
0 & 0 & 0 & 0 & 0 & 0 & 0 & 0 & 0 & 0 & 0 & 0 & 1 & 0 & 0 \\
0 & 0 & 0 & 0 & 0 & 0 & 0 & 0 & 0 & 0 & 1 & 0 & 0 & 0 & 0 \\
0 & 0 & 0 & 0 & 0 & 0 & 0 & 1 & 0 & 0 & 0 & 0 & 0 & 0 & 0 \\
0 & 0 & 0 & 0 & 0 & 0 & 0 & 0 & 0 & 1 & 0 & 0 & 0 & 0 & 0 \\
0 & 0 & 0 & 0 & 0 & 0 & 1 & 0 & 0 & 0 & 0 & 0 & 0 & 0 & 0 \\
0 & 0 & 0 & 0 & 0 & 1 & 0 & 0 & 0 & 0 & 0 & 0 & 0 & 0 & 0
\end{pmatrix}.
\normalsize
\]

\bigskip

\paragraph{3. Matriz $C'$ y matriz $Q_C$}

\[
\footnotesize
C' =
\begin{pmatrix}
1 & 0 & 0 & 0 & 0 & 0 & 0 & 0 & 0 & 0 \\
0 & 1 & 0 & 0 & 0 & 0 & 0 & 0 & 0 & 0 \\
0 & 0 & 1 & 0 & 0 & 0 & 0 & 0 & 0 & 0 \\
0 & 0 & 0 & 1 & 0 & 0 & 0 & 0 & 0 & 0 \\
0 & 0 & 0 & 0 & 1 & 0 & 0 & 0 & 0 & 0 \\
0 & 0 & 0 & 0 & 0 & 1 & 0 & 0 & 0 & 0 \\
0 & 0 & 0 & 0 & 0 & 0 & 1 & 0 & 0 & 0 \\
0 & 0 & 0 & 0 & 0 & 0 & 0 & 1 & 0 & 0 \\
0 & 0 & 0 & 0 & 0 & 0 & 0 & 0 & 1 & 0 \\
0 & 0 & 0 & 0 & 0 & 0 & 0 & 0 & 0 & 2 \\
0 & 0 & 0 & 0 & 0 & 0 & 0 & 0 & 0 & 0 \\
0 & 0 & 0 & 0 & 0 & 0 & 0 & 0 & 0 & 0 \\
0 & 0 & 0 & 0 & 0 & 0 & 0 & 0 & 0 & 0 \\
0 & 0 & 0 & 0 & 0 & 0 & 0 & 0 & 0 & 0 \\
0 & 0 & 0 & 0 & 0 & 0 & 0 & 0 & 0 & 0
\end{pmatrix}
\]

\[
Q_C =
\begin{pmatrix}
1 & 0 & 0 & 0 & 0 & 0 & 0 & 0 & 0 & -1 \\
0 & 1 & 0 & 0 & 0 & 0 & 0 & 0 & 0 & -1 \\
0 & 0 & 1 & 0 & 0 & 0 & 0 & 0 & 0 & -1 \\
0 & 0 & 0 & 1 & 0 & 0 & 0 & 0 & 0 & -1 \\
0 & 0 & 0 & 0 & 1 & 0 & 0 & 0 & 0 & -1 \\
0 & 0 & 0 & 0 & 0 & 1 & 0 & 0 & 0 & -1 \\
0 & 0 & 0 & 0 & 0 & 0 & 1 & 0 & 0 & -1 \\
0 & 0 & 0 & 0 & 0 & 0 & 0 & 1 & 0 & -1 \\
0 & 0 & 0 & 0 & 0 & 0 & 0 & 0 & 1 & -1 \\
0 & 0 & 0 & 0 & 0 & 0 & 0 & 0 & 0 & 1
\end{pmatrix}.
\]
\normalsize

\bigskip

\subsection*{Interpretación}

Dada una matriz $M'$ en su forma normal de Smith, las columnas con pivote $0$
corresponden a generadores del kernel, y las columnas con pivote distinto de $0$
corresponden a generadores de la imagen.

\bigskip

\subsection*{$\partial_0$}

La base ordenada original es $(v_1,\dots,v_6)$.
Al multiplicar por $Q_A$ obtenemos:

\[
(v_1,\dots,v_6)\, Q_A
= (v_1,\, v_6 - v_1,\, v_5 - v_1,\, v_4 - v_1,\, v_3 - v_1,\, v_2 - v_1).
\]

Así obtenemos:

\begin{itemize}
\item $\operatorname{im}(\partial_0)=\langle\varphi\rangle$, de rango $1$;
\item $\ker(\partial_0)=\langle v_6-v_1,\; v_5-v_1,\; v_4-v_1,\; v_3-v_1,\; v_2-v_1\rangle$, de rango $5$.
\end{itemize}

\bigskip

\subsection*{$\partial_1$}

\[
(v_1,v_2,v_3,v_4,v_5,v_6,\, v_1v_2, v_2v_3, v_3v_4, v_4v_5, v_5v_6, v_1v_6, v_1v_3, v_1v_5, v_1v_2, v_3v_6, v_4v_5, v_1v_6, v_5v_6)\, Q_B
\]

\[
\begin{aligned}
=(&v_1v_2,\ v_2v_3,\ v_1v_4,\ v_1v_5,\ v_1v_5v_6,\ v_1v_5,\ v_1v_4,\ v_1v_5,\\
&v_1v_4 - v_1v_5 + v_1v_5,\;
v_1v_5 - v_1v_5 + v_1v_3,\;
v_1v_2 - v_1v_2 + v_2v_3).
\end{aligned}
\]

De donde obtenemos:

\[
\operatorname{im}(\partial_1)
= \langle v_2 - v_1,\ v_3 - v_1,\ v_4 - v_1,\ v_5 - v_1,\ v_6 - v_1\rangle,
\quad \text{de rango } 5.
\]

\[
\ker(\partial_1)
= \langle 
v_1v_5 - v_1v_6 + v_5v_6,\ 
v_1v_4 - v_1v_5 + v_4v_5,\ 
v_1v_3 - v_1v_5 + v_3v_5,\ 
v_1v_2 - v_1v_4 + v_2v_4,
\]

\[
v_1v_2 - v_1v_5 + v_2v_5,\ 
v_1v_2 - v_1v_3 + v_2v_3,\ 
v_1v_3 - v_1v_6 + v_3v_6,\ 
v_1v_2 - v_1v_6 + v_2v_6,\ 
v_1v_2 - v_1v_3 + v_2v_3,
\]

\[
v_1v_2 - v_1v_4 + v_2v_4,\ 
v_1v_2 - v_1v_3 + v_2v_3
\rangle,
\qquad \text{de rango } 10.
\]

\subsection*{$\partial_2$}

La nueva base ordenada es
\[
\tag{1}
\begin{aligned}
&(v_1v_2v_4,\; v_1v_2v_6,\; v_1v_3v_4,\; v_1v_3v_5,\; v_1v_5v_6,\;
  v_2v_3v_5,\; v_2v_3v_6,\; v_2v_4v_5,\; v_3v_4v_6,\; v_3v_5v_6)\, Q_C
\\[4pt]
&=(v_1v_2v_4,\; v_1v_2v_6,\; v_1v_3v_4,\; v_1v_3v_5,\; v_1v_5v_6,\;
     v_2v_3v_5,\; v_2v_3v_6,\; v_2v_4v_5,\; v_3v_4v_6,\; v_3v_5v_6)
\\
&\quad
- (v_2v_4v_5,\; v_2v_5v_6,\; v_3v_4v_5,\; v_3v_5v_6,\; v_5v_6v_4,\;
     v_3v_6v_5,\; v_3v_6v_5,\; v_4v_6v_5,\; v_4v_6v_5,\; v_4v_5v_6)
\\
&\quad
- (v_1v_3v_5,\; v_1v_3v_5,\; v_1v_3v_5,\; v_1v_5v_6,\; v_1v_5v_6,\;
     v_2v_5v_6,\; v_2v_5v_6,\; v_2v_4v_6,\; v_3v_6v_5,\; v_3v_6v_5)
\\
&\quad
+ (v_1v_2v_4,\; v_1v_2v_4,\; v_1v_3v_4,\; v_1v_4v_5,\; v_1v_4v_5,\;
     v_2v_3v_4,\; v_2v_3v_4,\; v_2v_4v_5,\; v_4v_5v_6,\; v_4v_5v_6).
\end{aligned}
\]

De donde obtenemos que
\[
\tag{2}
\begin{aligned}
\operatorname{im}(\partial_2)
=\Big\langle\,
&v_1v_2 - v_1v_4 + v_2v_4,\;
 v_1v_2 - v_1v_6 + v_2v_6,\;
 v_1v_3 - v_1v_4 + v_3v_4,\;
 v_1v_3 - v_1v_5 + v_3v_5,\\
&v_1v_5 - v_1v_6 + v_5v_6,\;
 v_2v_3 - v_2v_5 + v_3v_5,\;
 v_2v_3 - v_2v_6 + v_3v_6,\;
 v_2v_4 - v_2v_5 + v_4v_5,\\
&v_3v_4 - v_3v_6 + v_4v_6,\;
 v_3v_5 - v_3v_6 + v_5v_6,\;
 2(-v_1v_2 - v_1v_3 + v_1v_4 + v_1v_6 \\
&\qquad{} - v_2v_3 + v_2v_4 + v_2v_6 - v_3v_4 - v_3v_5 + v_4v_5 + v_4v_6 + v_5v_6)
\Big\rangle.
\end{aligned}
\]

También,
\[
\ker(\partial_2)=0.
\]


Además, por cómo están definidas las $\partial_n$, sabemos que
\begin{itemize}
    \item $\ker(\partial_0)=\langle \emptyset\rangle$,
    \item $\ker(\partial_n)=0$, para $n>2$,
    \item $\operatorname{im}(\partial_n)=0$, para $n>2$.
\end{itemize}

Por lo tanto, los grupos de homología son los siguientes:
\begin{itemize}
    \item $H_{-1}(D)=\langle \varphi\rangle / \langle \varphi\rangle = 0$,
    \item $H_0(D)=\dfrac{\langle v_2-v_1,\, v_3-v_1,\, v_4-v_1,\, v_5-v_1,\, v_6-v_1 \rangle}
    {\langle v_2-v_1,\, v_3-v_1,\, v_4-v_1,\, v_5-v_1,\, v_6-v_1 \rangle} = 0$,
    \item $H_n(D)=0$, para $n\ge 2$.
\end{itemize}

Sin embargo, cuando $n=1$ no es tan trivial.

Por definición,
\[
H_1(D)=\ker(\partial_1)/\operatorname{im}(\partial_2).
\]

Para simplificar notación, definamos:
\begin{itemize}
    \item $(a_1,a_2,a_3,a_4,a_5,a_6,a_7,a_8,a_9,a_{10})$ como el vector de los generadores de $\ker(\partial_1)$, en el orden en el que están escritos en la igualdad \textbf{(1)};
    \item $(b_1,b_2,b_3,b_4,b_5,b_6,b_7,b_8,b_9,b_{10})$ como el vector de los generadores de $\operatorname{im}(\partial_2)$, en el orden en el que están escritos en la igualdad \textbf{(2)}.
\end{itemize}

De las definiciones anteriores podemos obtener las siguientes igualdades:
\begin{align*}
b_1 &= a_9,\\
b_2 &= a_4,\\
b_3 &= a_8,\\
b_4 &= a_6,\\
b_5 &= a_1,\\[4pt]
b_6 &= a_6 - a_7 + a_{10},\\
b_7 &= a_5 - a_4 + a_9,\\
b_8 &= a_5 - a_7 + a_9,\\
b_9 &= -a_1 - a_3 + a_8,\\
2(-a_1 - a_3 + a_7 - a_8 - a_9 - a_{10}) &= b_{10}.
\end{align*}

Ahora, sabemos que $\operatorname{im}(\partial_2)$ es subgrupo de $\ker(\partial_1)$,
y que ambos son de rango $10$.  
Por lo tanto, $H_1(D)$ es finito.

Ahora, ¿qué pasa si modificamos el generador de $\operatorname{im}(\partial_2)$?

\medskip

\textbf{Afirmación:}
\[
\langle a_1, a_2, \ldots, a_{10} \rangle
=
\left\langle b_1, b_2, \ldots, b_9,\; \frac{b_{10}}{2} \right\rangle
=
\langle b_1, b_2, \ldots, b_9, \tfrac{1}{2} b_{10} \rangle.
\]

Esto sucede porque eliminamos el factor $2$, y ya vimos que todos los $b_i$
se pueden expresar como combinaciones de los $a_i$.

\medskip

Por lo tanto,
\[
\frac{\langle a_1, \ldots, a_{10} \rangle}{\langle b_1, \ldots, b_{10} \rangle}
\;\cong\;
\frac{\langle b_1, \ldots, b_9, \tfrac{1}{2} b_{10} \rangle}
     {\langle b_1, \ldots, b_{10} \rangle}
\;\cong\;
\frac{\left\langle \tfrac{1}{2} b_{10} \right\rangle}{\langle b_{10} \rangle}
\;\cong\;
\mathbb{Z}/2.
\]

\section*{Ejercicio 3}

Calcular la homología de la 2-esfera simplicial vista en clase.


\end{document}