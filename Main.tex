\documentclass[12pt]{book}
\usepackage{amsfonts}
\usepackage{amsmath}      % Para \numberwithin
\usepackage{amsthm} 
\usepackage{hyperref}
\usepackage{url}
\theoremstyle{plain}
\numberwithin{equation}{section} %change this to make globally numbered equations
\newtheorem{thm}{Theorem}[section] %remove [section] to make globally numbered environments
\usepackage[spanish]{babel}
\newtheorem{theorem}[thm]{Theorem}
\newtheorem{lemma}[thm]{Lemma}
\newtheorem{example}[thm]{Example}
\newtheorem{definition}[thm]{Definición}
\newtheorem{proposition}[thm]{Proposicion}
\newtheorem{corollary}[thm]{Corollary}
\newtheorem{remark}[thm]{Observación}
\newtheorem{notation}[thm]{Notación}


% Enumeración 
\renewcommand{\thechapter}{\Roman{chapter}}  % Romana para los capítulos

%Las secciones y subsecciones en números arábigos
\renewcommand{\thesection}{\arabic{section}}
\renewcommand{\thesubsection}{\thesection.\arabic{subsection}}
% Comandos útiles
\newcommand{\deftext}[1]{\textbf{\textit{#1}}}
%%%%%%%%%%%%%%%%%%%%%%%%%%%%%%%%%%%%%%%%%%%%%%%%%%%%%%

\begin{document}
\author{Sebastián Rodríguez Labastida, Álvaro Matanzo Hermoso}
\title{Análisis Topológico de Datos}
\date{\today}
\frontmatter
\maketitle
\tableofcontents
\mainmatter
\chapter{Preludio de Algebra}
\section{Grupos}
\begin{definition}\label{def:grupo}
Un \deftext{grupo}\footnote{Por convención, en este texto todos los grupos serán abelianos.} es un conjunto $A$ junto con una operación $+\colon A\times A\rightarrow A$ que satisface que para cualesquiera $a,b,c \in A$ se cumple
\begin{enumerate}
	\item $a+(b+c) \ =\ (a+b)+c$.
	\item $a+b \ =\ b+a $.
	\item Existe $0_A \in A$ que satisface que para toda $z\in A$ se cumple que $z+0_A \ =\ a$.
	\item Existe $x\in A$ tal que $a+x \ =\ 0_A$.
\end{enumerate}
Cuando no exista ambiguedad usaremos el símbolo $0$ sin mencionar a que grupo pertenece.
\end{definition}

\begin{definition}
Sea $\{B_i\}_{i\in I}$ una familia de grupos. Un \emph{vector}
\[
(\ldots, b_i, \ldots)
\]
es una familia que asigna a cada índice $i \in I$ un elemento $b_i \in B_i$.
También puede verse como una función
\[
f : I \longrightarrow \bigcup_{i\in I} B_i
\]
tal que
\[
f(i) = b_i \in B_i \quad \text{para cada } i\in I.
\]

La igualdad y la suma de vectores se definen \emph{coordenada a coordenada}:
\[
(b_i)_{i\in I} = (b_i')_{i\in I}
\quad\Longleftrightarrow\quad
b_i = b_i' \ \text{para todo } i\in I,
\]
\[
(b_i)_{i\in I} + (b_i')_{i\in I}
= (\,b_i + b_i'\,)_{i\in I}.
\]

El conjunto de todos estos vectores, con la operación descrita, forma un grupo que se denota por
\[
C = \prod_{i\in I} B_i,
\]
y se llama el \emph{producto directo} o \emph{producto cartesiano} de los grupos $B_i$.
\end{definition}


\begin{definition}
Sean $\{B_i\}_{i \in I}$ grupos abelianos. Se llama \emph{suma directa} de los grupos $B_i$ al subgrupo del producto directo
\[
\bigoplus_{i \in I} B_i 
= 
\left\{
(b_i)_{i \in I} \in \prod_{i \in I} B_i 
\;\middle|\;
b_i = 0 \text{ salvo para un número finito de índices } i
\right\},
\]
donde la operación de grupo está dada componente a componente.

A los homomorfismos de inclusión
\[
\rho_i : B_i \longrightarrow \bigoplus_{i \in I} B_i,
\qquad
b_i \longmapsto (\ldots,0,b_i,0,\ldots),
\]
se les llama \emph{inyecciones coordenadas}. 

El grupo $\bigoplus_{i \in I} B_i$ se denomina también \emph{suma directa externa} de los $B_i$ y puede verse como el subgrupo del producto directo formado por los elementos con soporte finito.
\end{definition}

% ============================================================
% Definiciones para issue #17: grupo abeliano libre, tensor y Tor
% Basadas en Fuchs, Abelian Groups (Springer, 2015)
% ============================================================

\begin{definition}\label{def:free-abelian-group}
Un \deftext{grupo abeliano libre} es una suma directa de grupos cíclicos infinitos.
Si estos grupos cíclicos son generados por elementos $x_i$ $(i \in I)$, entonces
el grupo libre será
\[
F  \ = \  \bigoplus_{i \in I} \langle x_i \rangle.
\]
El conjunto $\{x_i\}_{i \in I}$ es una \textit{base} de $F$. Los elementos de $F$ son
combinaciones lineales finitas de la forma
\[
g \ =\ n_1x_{i_1} + \cdots + n_kx_{i_k}, \quad \text{con } n_j \in \mathbb{Z},
\]
Dos combinaciones representan el
mismo elemento de $F$ si y solo si difieren en el orden de los términos.
La suma se define añadiendo los coeficientes de los mismos generadores.
En particular, $F$ está determinado, salvo isomorfismos, por la cardinalidad
de sus generadores, al cual se llama el \textit{rango} del grupo libre.
\end{definition}

\begin{theorem}[Propiedad universal de los grupos libres]
Sea $X$ un conjunto libre de generadores del grupo libre $F$. 
Toda función
\[
f : X \longrightarrow A
\]
con valores en un grupo abeliano $A$ se extiende de manera única a un homomorfismo
\[
\varphi : F \longrightarrow A.
\]
Esta propiedad caracteriza a los conjuntos libres de generadores, y por lo tanto a los grupos libres.
\end{theorem}


\begin{definition}\label{def:tensor-product}
Sean $A$ y $C$ grupos abelianos. Sea $X$ el grupo libre sobre el conjunto
$A \times C$, cuyos generadores son $(a,c)$ con $a \in A$ y $c \in C$.
Sea $Y$ el subgrupo de $X$ generado por todos los elementos de la forma
\[
(a_1 + a_2, c) - (a_1, c) - (a_2, c), \quad
(a, c_1 + c_2) - (a, c_1) - (a, c_2),
\]
para todo $a,a_1,a_2 \in A$ y $c,c_1,c_2 \in C$. El \textit{producto tensorial}
de $A$ y $C$ se define como el cociente
\[
A \otimes C \ =\ X / Y.
\]
Dado $(a,c) \in A \times C$ definimos el \textit{tensor} de $a$ con $c$, $a \otimes c$, como la clase del generador $(a,c)$; entonces los elementos de
$A \otimes C$ son combinaciones lineales finitas de tales tensores, sujetas a
las relaciones
\[
(a_1 + a_2) \otimes c \ =\ a_1 \otimes c + a_2 \otimes c, \qquad
a \otimes (c_1 + c_2) \ =\ a \otimes c_1 + a \otimes c_2.
\]
El grupo $A \otimes C$ cumple la \textit{propiedad universal}: para toda
función bilineal $g : A \times C \to G$, con $G$ un grupo abeliano, existe un único
morfismo $\overline{g} : A \otimes C \to G$ tal que
$\overline{g}(a \otimes c) = g(a,c)$. Esta propiedad caracteriza a
$A \otimes C$ de manera única, salvo isomorfismos.
\end{definition}

\begin{definition}
La parte de torsión de un grupo $A$ es
\[
t(A) := \{\, x \in A \mid nx = 0 \text{ para algún } n \in \mathbb{N} \,\}.
\]
Si $t(A)=A$, decimos que $A$ es de torsión.
En este caso,
\[
A = \bigoplus_{p} A_p.
\]
\end{definition}

\begin{theorem}[Teorema fundamental de los grupos finitamente generados]
Todo grupo $G$ tal que existe  $X \in [G]^{<\omega}$\footnote{En general, si $A$ es un conjunto, entonces $[A]^{<\omega}$ representa la colección de todos los subconjuntos finitos de $A$. } con
$\langle X \rangle = G$ es isomorfo a la suma directa de un grupo libre finitamente 
generado y un número finito de grupos cíclicos de orden $p^{k}$, con $p$ primo 
y $k \ge 1$. Esta descomposición es única salvo isomorfismos. En particular,
\[
G = F(G) \oplus t(G),
\]
donde $F(G)$ es la parte libre de $G$ y $t(G)$ es su parte de torsión.
\end{theorem}

\begin{definition}\label{def:tor}
El \deftext{producto de torsión} de dos grupos abelianos $A$ y $C$, denotado por
$\mathrm{Tor}(A,C)$, se define como el grupo abeliano libre generado por las tercias
$(a,m,c)$ con $a \in A$, $c \in C$ y $m \in \mathbb{N}$ tales que $ma = 0 = mc$,
sujetos a las relaciones
\begin{align*}
(a_1 + a_2, m, c) &= (a_1, m, c) + (a_2, m, c),\\
(a, m, c_1 + c_2) &= (a, m, c_1) + (a, m, c_2),\\
(a, mn, c) &= (na, m, c) = (a, m, nc).
\end{align*}
Con estas relaciones, el grupo $\operatorname{Tor}(A,C)$ es abeliano y satisface una
\textit{simetría natural}
\[
\operatorname{Tor}(A,C) \cong \operatorname{Tor}(C,A).
\]
\end{definition}

\begin{proposition}
Sea $\mathcal{G}$ una familia de grupos abelianos. Entonces,
\[
\left(\bigoplus_{G\in\mathcal{G}} G\right) \otimes C 
\;\cong\;
\bigoplus_{G\in\mathcal{G}} (G \otimes C).
\]
\end{proposition}


\begin{proposition}
Si $A$ y $C$ son finitos, entonces
\[
A \otimes C \;\cong\; \operatorname{Tor}(A, C),
\]
y ambos funtores son conmutativos; es decir,
\[
A \otimes C \cong C \otimes A 
\quad\text{y}\quad 
\operatorname{Tor}(A,C) \cong \operatorname{Tor}(C,A).
\]
\end{proposition}


% ============================================================

\chapter{Complejos}
\section{Complejos de Cadenas}
\begin{definition}[Complejo de Cadenas]\label{def:CC}
Un \deftext{complejo de cadenas} es una familia $\mathcal{K} :=\{ (K_n, \partial_n) \colon n\in \mathbb{Z}\}$ tal que para toda $n \in \mathbb{Z}$ se satisfacen las siguientes condiciones
\begin{enumerate}
	\item $K_n$ es un grupo abeliano;
	\item $\partial_n \colon K_{n+1} \to K_n$ es morfismo de grupos abelianos; y
	\item $\partial_n \partial_{n+1} = 0$\label{eq:partC}. 
\end{enumerate}
\end{definition}
\begin{remark}
La ecuación última es equivalente a decir que para toda $n\in \mathbb{Z}$ se cumple que
\begin{align*}
	\operatorname{im}(\partial_{n+1} ) \leq \ker(\partial_n).
\end{align*}
\end{remark}
\begin{definition}
Dado un complejo de cadenas $\mathcal{K}$ definimos la \deftext{Homología} de $\mathcal{K}$, en símbolos $H(\mathcal{K})$, como la familia de grupos abelianos $\{ H_n(\mathcal{K}}) \mid n \in \mathbb{Z} \}$, donde
\begin{center}
	$H_n(\mathcal{K})\ :=\ \ker(\partial_n) / \operatorname{im}(\partial_{n+1})$.
\end{center} 
Nos referimos a los elementos de $K_n$ como \deftext{$n$-cadenas}, a los de $\ker(\partial_n)$ como \deftext{$n$-cíclos} y a los de $\operatorname{im}(\partial_n)$ como \deftext{$n$-fronteras}. Cuando no exista confusión escribiremos la clase lateral $c + \operatorname{im}(\partial_{n+1}) \in H_n(\mathcal{K})$ como $[c]$. Además, si $c' \in [c]$, diremos que $c'$ y $c$ son \deftext{homólogos}.
\end{definition}

\begin{definition}[Morfismo de cadenas]
Dados dos complejos de cadenas $\mathcal{K}$ y $\mathcal{K}'$ un \deftext{morfismo de cadenas} $f$, denotado por $f \colon \mathcal{K}\to\mathcal{K}'$ es una familia de morfismos $\{f_n \colon K_n \to K_n' \mid n\in \mathbb{Z}\}$ de tal manera que $\partial_{n+1}' f_{n+1} = f_n \partial_n$.
\end{definition}

\begin{proposition}
\begin{enumerate}
	\item El morfismo trivial de grupos\footnote{Aquí nos referimos al morfismo que manda todo elemento de un grupo al $0$ de otro.} abelianos induce de manera natural un morfismo de cadenas.
	\item Existe $1_\mathcal{K}:\mathcal{K}\to \mathcal{K}$, con $1_{\mathcal{K}_n} = \text{Id}_{K_n}$ la identidad.
	\item S $\mathcal{K}$, $\mathcal{J}$ y $\mathcal{L}$ son complejos de cadenas y $f\colon \mathcal{K}\to \mathcal{J}$ y $g \colon \mathcal{J}\to \mathcal{L}$ son morfismos de cadenas, entonces la familia de morfismos $\{ g_n \circ f_n \mid n\in \mathbb{Z} \}$ es el morfismo de cadenas $g\circ f \colon \mathcal{K}\to \mathcal{L}$.
\end{enumerate}
2 y 3 en suma nos dicen que existe la categoría de complejos de cadenas. 
\end{proposition}
\begin{proposition}
Dados dos complejos de cadenas $\mathcal{K}$ y $\mathcal{L}$ y un morfismo de cadenas $f:\mathcal{K} \to \mathcal{L}$, se tiene que la familia de morfismos $A$, $\{H_n(f) : n\in \mathbb{Z} \}$ donde el morfismo $H_n(f):H_n(\mathcal{K})\ \to H_n(\mathcal{L})$ está dado por
\begin{center}
$H_n(f)(c + \operatorname{im}\partial_{n+1}) = f(c) + \operatorname{im}\partial_{n+1}$,
\end{center}
es un morfismo de grupos abelianos. 
\end{proposition}
\section{Complejos Simpliciales}
\begin{definition}
Sea $C_0$ un conjunto finito. Un \deftext{Complejo Simplicial} sobre $C_0$ es una familia $\mathcal{C}\subseteq P(C_0)$ cerrada bajo subconjuntos, es decir que si $\sigma \in \mathcal{C}$ y $\tau \subseteq \sigma$, entonoces $\tau \in \mathcal{C}$.

\end{definition}
\begin{definition}
Un complejo simplicial $C=P(C_0)$ se llama \deftext{Simplejo}. En este caso $C_0$ es la única faceta de $P(C_0)$. En este caso escribimos $\overline{C_0}$ en lugar de $C$.
\end{definition}
\begin{remark}
El complejo simplicial generado por $F\subseteq P(V_0)$ es el mínimo complejo simplicial $K$ de manera que $F\subseteq K$. Una manera de hacer esto es $\bigcup F = \bigcup_{\sigma \in F} \overline{\sigma}$.
\end{remark}
\begin{notation}\label{not:ident}
Si $a\in C_0$ y $C$ es un complejo simplicial sobre $C_0$, con $\{a\}\in C$, identificamos el símbolo $a$ con $\{a\}$. En general, pero no siempre, identificaremos la cadena de símbolos $a_1 a_2 \cdots a_n$ con el conjunto $\{a_1, a_2, \ldots, a_n\}$.
\end{notation}

\begin{definition}[n-esfera]
Sea $\{a_n \mid n<\omega \}$ un conjunto de símbolos distintos por pares; definimos la \deftext{$0$-esfera} como $S_{\Delta}^0(a_0) := \{ \emptyset, (a_0,0), (a_0, 1)\}$ y para toda $n < \omega$ definimos la \deftext{$(n+1)$-esfera} como
\[
	S_{\Delta}^{n+1}(a_0, \ \ldots,\ a_{n+1}):=  S_{\Delta}^{n+1}(a_0, \ \ldots,\ a_{n})*S_{\Delta}^0(a_{n+1});
\]
donde $*$ es el ensamble de complejos simpliciales definido como sigue: Si $C$ y $D$ son complejos sin vértices en común, entonces el esnamble de $C$ con $D$ es el complejo
\[
C*D := \left\{\sigma \cup \tau \mid (\sigma, \tau) \in C\times D \right\}.
\]
\end{definition}

\subsection*{Funciones Simpliciales}
Para todo complejo simplicial $C$ denotamos con $C_0$ al conjunto de vértices de $C$. Sean $D$ y $C$ dos complejos simpliciales. 
Una función $f\colon C_0 \to D_0$ es una \deftext{función simplicial} si para todo $\sigma \in C$ se cumple que $f[\sigma] \in D$.

\begin{remark}
Dada una función simplicial $f\colon C_0 \to D_0$, y recordando la convención hecha en \ref{not:ident} podemos pensar que $f$ está definida en todo $C$, de manera que siempre asumiremos que $f\colon C \to D$ para una función simplicial. En este caso $f(\sigma) := f[\sigma]$. 
\end{remark}
\subsection*{Realización Geométrica}
\begin{definition}
Un conjunto finito $\{x_i \mid i\leq m\}$ en $\mathbb{R}^n$ es \deftext{afinmente independiente} si siempre que 
\[
	\sum_{i\in J} t_i x_i = \sum_{i\in J} s_i x_i
\]
con $J\subseteq \{0,\ \ldots, \ m\}$ y $\sum_{i\in J} t_i = \sum_{i\in J} s_i = 1$ y $t_i , s_i \geq 0$ para toda $i \in J$, entonces $t_i = s_i$ para toda $i\in J$.\\
\noident Dado un conjunto afinmente independiente $A$, una combinación afin de $A$ es un vector de la forma $\sum_{a\in A}S_A a$, con $\sum_{a\in A}S_A =1$ y $S_a \geq 0$.
\end{definition}
\begin{remark}
$A\subseteq \mathbb{R}^n$ es afinmente independiente si y sólo si $A\setminus \{a\}-a$ es linealmenete independiente para algun $a\in A$.
\end{remark}
\begin{definition}
Sea $A$ un conjunto afinmente independiente. La \deftext{cápsula convexa} de $A$ es el conjunto de todas las combinaciones afines de $A$.
\end{definition}

\begin{definition}
Sea $V$ un conjunto finito, el \deftext{$V$-simplejo estandar} es la cápsula convexa de $V$ en $\mathbb{R}^n$.
\end{definition}
\begin{definition}
Sea $C$ un complejo simplicial con vértices $C_0$. La \deftext{realización geométrica estandar} de $C$ es la unión de los $\sigma$-simplejos estandar en $\mathbb{R}^{C_0}$, con $\sigma \in C$, con la topología de subespacio.\\
\noindent Todo espacio topológico isomorfo a la realización geométrica estandar de $C$ será llamado realización geométrica de $C$ y sin miramientos nos referiremos a cualquiera de ellos como $|C|$.
\end{definition}
\subsection*{Complejos Simpliciales Geométricos}
Un $n$-simplejo geométrico es la cápsula convexa de un conjunto afinmente independiente de cardinalidad $n+1$. Una cara de un simplejo geométrico $\sigma$ generado por $A$ es un simplejo generado por un subconjunto de $A$.\\
\noindent A cada $n$-simplejo geométrico generado por $A$ le asociamos el $n$-simplejo(abstracto) $P(A)$. Por otro lado, todo $n$-simplejo $\sigma$ tiene asociado un $n$-simplejo geométrico $|\sigma |$.\\
\noindent Un complejo simplicial geométrico es un conjunto de simplejos geométricos, cerrado bajo curvas, tal que la intersección de cualesquiera dos de ellos tiene que ser una cara de ambos.\\

Sean $C$ y $D$ dos complejos simpliciales y $f\colon C \to D$ una función simplicial. Consideremos las respectivas realizaciones geométricas $|C| \subseteq \mathbb{R}^{C_0}$ y $|D| \subseteq \mathbb{R}^{D_0}$. De este modo, todo elemento de $|C|$ es de la forma $\sum_{a\in \sigma} t_a a$ con $\sigma \in C$.
De esta manera, podemos definir a la función $|f|\colon |C| \to |D|$ como
\[
	|f|(\sum_{a\in \sigma}t_a a) := \sum_{a\in \sigma}t_a f(a)
\]
para toda $\sigma$ y toda combinación lineal afin. 
\begin{remark}
$a\in \mathbb{R}^{C_0}$ es la función definida como 
\[
	a(x) := \begin{cases}
	0 & a\neq x,\\
	a & a = x. 
	\end{cases}
\]
\end{remark}


\end{document}